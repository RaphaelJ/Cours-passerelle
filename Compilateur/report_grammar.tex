\documentclass[a4paper]{article}
\usepackage[utf8]{inputenc}
\usepackage[francais]{babel}
\usepackage[babel=true]{csquotes}
\usepackage{graphicx}
\usepackage{listings}

\pagestyle{headings}

\title{Compilateur : Grammaire}
\author{Raphaël Javaux (groupe \#14)}
\date{}

\begin{document}

\maketitle

    \paragraph{}J'ai choisi d'implémenter un langage à typage statique reprenant
les fonctionnalités de base du langage C nécessaire à l'expression du QuickSort.

    \paragraph{Fonctionnalités minimales}

    \begin{itemize}
        \item Types de base : $int$, $bool$, $float$ ainsi que des tableaux de 1
à N dimensions de ces premiers. Ces types seront directement symétriques
dans leurs représentations à ceux du langage C ;
        \item Qualifieur de type $const$ pour déclarer des variables constantes
;
        \item Déclaration, définition et appel de fonction (éventuellement
récursif), opérateur $return$
;
        \item Opérateurs arithmétiques ($+$, $-$, $*$, $/$, $\%$), relationnels 
($==$, $!=$, $<$, $>$, $<=$, $>=$) et logiques ($\&\&$, $||$) avec respect des
règles de priorité et d'associativité ;
        \item Structures de contrôle $if$ et $while$.
    \end{itemize}

    \paragraph{}Les fonctionnalités exprimées ci-dessus sont les fonctionnalités 
élémentaires nécessaires pour permettre l'experession du QuickSort.
Une fois qu'une version fonctionnelle du compilateur sera réalisée, d'autres
fonctionnalités y seront éventuellement rajoutées, comme la possibilité de
déclarer des types composites ($struct$), la gestion des fonctions imbriquées,
l'inférence des types des variables lorsqu'elles sont directement initialisées,
\dots

    \paragraph{Exemple}QuickSort exprimé à l'aide de ce langage
:

\begin{lstlisting}
void swap(int[], int, int);

void qsort(int[] v, int left, int right)
{
    if (left >= right) {
        return;
    }

    int i = left + 1;
    int last = left;
    while (i <= right) {
        if (v[i] < v[left]) {
            last = last + 1;
            swap(v, last, i);
        }
        i = i+1;
    }

    swap(v, left, last);
    qsort(v, left, last - 1);
    qsort(v, last + 1, right);
}

void swap(int[] v, int i, int j)
{
    int tmp = v[i];
    v[i] = v[j];
    v[j] = v[i];
}
\end{lstlisting}

    \paragraph{Choix} Si ce langage est un peu plus compliqué à parser qu'un
langage basé sur des S-Expressions comme Lisp ou Scheme, son typage statique
ainsi que l'absence de garbage collector rendra plus simple la génération du
code LLVM.

De plus, étant donné que les conventions d'appels et les types seront
symétriques à ceux du langage C, il sera possible de simplement linker un
fichier objet écrit en C avec un autre fichier objet écrit à l'aide de ce
langage\footnote{J'envisage de simplement linker la fonction $qsort$ avec un
programme C permettant de la tester plutôt de d'implémenter les fonctions
d'entrée/sortie dans la version finale du travail, si cela est permis.}.

\end{document}