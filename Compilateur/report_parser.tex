\documentclass[a4paper]{article}
\usepackage[utf8]{inputenc}
\usepackage[francais]{babel}
\usepackage[babel=true]{csquotes}
\usepackage{graphicx}
\usepackage{listings}

\pagestyle{headings}

\title{Compilateur : Parseur}
\author{Raphaël Javaux (groupe \#14)}
\date{}

\begin{document}

\maketitle

  \section{Installation des dépendances}

    \paragraph{}Le compilateur est écrit en Haskell et doit être compilé avec
GHC. Le compilateur utilise Parsec, une librairie de combinateurs permettant de
réaliser facilement des parseurs.

    \paragraph{}Cabal un utilitaire permettant d'automatiser la compilation
et l'installation de programmes écrits en Haskell (similaire aux Ruby Gems).

    \paragraph{}GHC et cabal\footnote{Parfois nommé cabal-install} sont 
disponibles dans les dépots de la plupart des distributions Linux. Pour les
autres systèmes d'exploitation, un installateur combinant ces deux logiciels est
disponible à cette adresse : http://www.haskell.org/platform/.

    \paragraph{}Une fois GHC et cabal installés, il est possible d'installer les
dépendances du compilateur en se plaçant à la racine du projet et en exécutant
ces deux commandes :

\begin{lstlisting}
cabal update
cabal install --only-dependencies
\end{lstlisting}

  \section{Compilation}

    \paragraph{}Pour compiler le compilateur avec cabal, exécuter les deux
commandes suivantes :

\begin{lstlisting}
cabal configure
cabal build
\end{lstlisting}

    \paragraph{}L'exécutable compilé se nomme \textit{dist/build/coda/coda}.

  \section{Exécution}

    \paragraph{}Le compilateur accepte en entrée standard les sources et
retourne sur la sortie standard l'arbre syntaxique résultat du parsing. Exemple
d'utilisation pour obtenir l'arbre syntaxique correspondant au fichier source
contenant le Quicksort :

\begin{lstlisting}
./dist/build/coda/coda < quicksort.coda
\end{lstlisting}

\end{document}