\documentclass[a4paper]{article}
\usepackage[utf8]{inputenc}
\usepackage{graphicx}
\usepackage{verbatim}
\pagestyle{headings}

\title{Projet de base de données}
\author{Raphael Javaux et Marien Bourguignon}
\date{Avril 2013}

\begin{document}

\maketitle

\section{Technologies}
    \paragraph{}Nous avons décidé d'utiliser le framework bootstrap (une
librairie HTML5 + CSS3 + Javascript) pour la partie HTML de notre site web.
    Pour les accès à la base de données, nous avons choisi d'utiliser PDO
de PHP.

\section{Fonctionnement du Site Web}
    \paragraph{}Notre projet est accessible à l'adresse suivante :
    \begin{center}
        \textbf{http://ks201461.kimsufi.com/bdd/}
    \end{center}

      \subsection{Premiere connexion}
        \paragraph{}Lorsqu'un utilisateur se connecte pour la première fois, il 
est automatiquement redirigé sur la page de login (sigin.php).
    Si notre script arrive à se connecter à la base de données avec les 
identifiants qu'il a fourni, une variable de session sera initialisée, qui lui
évitera d'être redirigé à nouveau vers la page de connexion par la suite.

        \paragraph{}Il est possible de se connecter sur notre BDD 
avec l'utilisateur \textbf{ULg} et le mot de passe \textbf{batman}.

\section{Requêtes}
  \subsection{Fonctionnalité (a) (global\_search.php)}
    \paragraph{} Nous avons implémenté cette partie totalement génériquement.
Il est tout à fait possible de porter ce code dans une autre application sans
devoir changer la moindre parcelle de code.

    \paragraph{}Nous avons découpé le travail de la manière suivante:
    \begin{itemize}
        \item[1] Rechercher la liste des tables de la base de données utilisée ;
        \item[2] L'utilisateur sélectionne une table ;
        \item[3] Rechercher, pour la table sélectionnée, toutes les colonnes de
        la clef primaire ;
        \item[4] L'utilisateur spécifie les valeurs pour toutes les colonnes de
        la clé ;
        \item[5] On crée la requête finale, en prenant en compte le \textit{*} ;
        \item[6] Le résultat est affiché dans un tableau.
    \end{itemize}

        La procédure vérifie tous les champs fournis par l'utilisateur pour 
éviter tout trifouillage dans la base de données.

    \subsubsection{Requêtes}

    \paragraph{Recherche des tables}
    \begin{verbatim}
        SHOW TABLES
    \end{verbatim}

    \paragraph{Recherche des clefs primaires}
    \begin{verbatim}
        SHOW KEYS FROM <tablename> WHERE Key_name = 'PRIMARY'
    \end{verbatim}

    \paragraph{Recherche sur les clefs}
    \begin{verbatim}
        SELECT * FROM <tablename> WHERE <conditions>
        Ou <conditions> est du type "champ = valeur".
    \end{verbatim}

  \subsection{Fonctionnalité (b) (vehicule\_availability.php)}

    \paragraph{Sélection des informations des véhicules disponibles entre deux
    dates pour une catégorie}
    \begin{verbatim}
        SELECT v.*, m.*
        FROM vehicule AS v
        INNER JOIN modele AS m
            ON m.id_mod = v.id_mod
        WHERE m.id_cat = <id_cat>
          AND NOT EXISTS (
            -- Disponible s'il n'existe pas un contrat
            -- qui se superpose aux dates souhaitées.
            SELECT *
            FROM contrat AS c
            WHERE     <date_debut> <= c.date_fin
                  AND <date_fin>   >= c.date_debut
                  AND c.id_veh = v.id_veh
            );
    \end{verbatim}

  \subsection{Fonctionnalité (c) (new\_transaction.php)}

    \paragraph{}Les requêtes de cette page s'effectuent dans une même
transaction pour éviter des lecture non répétables entre la validité des
données fournies et l'insertion.

    \paragraph{Sélection de l'ensemble des noms des clients}
    \begin{verbatim}
        SELECT p.id_client,
               CONCAT(p.prenom_part, ' ', p.nom_part) AS nom
        FROM particulier AS p
        UNION
        SELECT s.id_client, s.nom_societe AS nom
        FROM societe AS s;
    \end{verbatim}

    \paragraph{Sélection de l'ensemble des conducteurs}
    \begin{verbatim}
        SELECT cc.no_permis,
               CONCAT(p.prenom_part, ' ', p.nom_part) AS nom
        FROM particulier AS p
        INNER JOIN cond_client AS cc
            ON cc.id_client = p.id_client
        UNION
        SELECT cc.no_permis, s.nom_societe AS nom
        FROM societe AS s
        INNER JOIN cond_client AS cc
            ON cc.id_client = s.id_client
        UNION
        SELECT ca.no_permis,
                CONCAT(ca.prenom_conducteur_autre, ' ', 
                        ca.nom_conducteur_autre) AS nom
        FROM cond_autre AS ca;
    \end{verbatim}

    \paragraph{Sélection des véhicules disponibles pour la période souhaitée}
    \begin{verbatim}
        SELECT v.*, m.*
        FROM vehicule AS v
        INNER JOIN modele AS m
            ON m.id_mod = v.id_mod
        WHERE NOT EXISTS (
            -- Disponible s'il n'existe pas un contrat
            -- qui se superpose aux dates souhaitées.
            SELECT *
            FROM contrat AS c
            WHERE   <date_debut> <= c.date_fin
                AND <date_fin>   >= c.date_debut
                AND c.id_veh = v.id_veh
            );
    \end{verbatim}

    \paragraph{Insertion du contrat}
    \begin{verbatim}
        INSERT INTO contrat
        (id_veh, id_client, date_debut, date_fin, km_debut, km_fin)
        VALUES (<id_veh>, <id_client>, <date_debut>, <date_fin>, NULL, NULL);
    \end{verbatim}

    \paragraph{Insertion des conducteurs du contrat}
    \begin{verbatim}
        INSERT INTO conduit (id_cont, no_permis)
        VALUES (<id_contrat>, <no_permis>);
    \end{verbatim}

  \subsection{Fonctionnalité (d) (finish\_contract.php)}

    \paragraph{}Les requêtes de cette page s'effectuent dans une même
transaction pour éviter de facturer deux fois la même commande.

    \paragraph{Sélection des contrats non facturés}
    \begin{verbatim}
        SELECT c.id_cont
        FROM contrat AS c
        WHERE NOT EXISTS (
            SELECT *
            FROM facture AS f
            WHERE f.id_cont = c.id_cont)
        ORDER BY c.id_cont;
    \end{verbatim}

    \paragraph{Mise à jour du kilométrage du contrat}
    \begin{verbatim}
        UPDATE contrat
        SET km_debut = <km_debut>, km_fin = <km_fin>
        WHERE id_cont = <id_cont>;
    \end{verbatim}

    \paragraph{Insertion et calcul de la facture}
    \begin{verbatim}
        INSERT INTO facture (id_cont, prix)
        SELECT con.id_cont,
                 DATEDIFF(con.date_fin, con.date_debut) * cat.prix_jour
               + (con.km_fin - con.km_debut) * cat.prix_km AS prix
        FROM contrat AS con
        INNER JOIN vehicule AS v
            ON v.id_veh = con.id_veh
        INNER JOIN modele AS m
            ON m.id_mod = v.id_mod
        INNER JOIN categorie AS cat
            ON cat.id_cat = m.id_cat
        WHERE con.id_cont = <id_cont>;
    \end{verbatim}

  \subsection{Fonctionnalité (e) (search\_id.php)}

    \paragraph{Sélection du véhicule ayant le plus voyagé sur une période}
    \begin{verbatim}
        SELECT c.id_veh, SUM(c.km_fin - c.km_debut) AS kms
        FROM contrat AS c
        WHERE c.date_debut >= <date_debut> AND c.date_fin <= <date_fin>
        GROUP BY c.id_veh
        ORDER BY kms DESC
        LIMIT 1;
    \end{verbatim}

\end{document}