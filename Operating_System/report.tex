\documentclass[a4paper]{article}
\usepackage[utf8]{inputenc}
\usepackage{graphicx}
\usepackage{verbatim}
\pagestyle{headings}

\title{Système d'exploitation : Projet 2}
\author{Raphael Javaux et Marien Bourguignon}
\date{Avril-Mai 2013}

\begin{document}

\maketitle

\section{Architecture}

 \subsection{Enregistrement d'une page dans le ZRLE cache}

    \paragraph{}Notre implémentation intercepte les pages allant être swappées
dans la fonction \texttt{swap\_writepage()}. Chaque page est analysée pour
calculer sa taille une fois compressée avec l'algorithme ZRLE.

    \paragraph{}Si la taille de la page une fois compressée est inférieure à
80\% à la taille d'une page non compressée, alors nous parcourons l'ensemble
des secteurs libres (listés sous la forme d'un chaînage) de nos "bins" à la
recherche du premier secteur suffisamment grand pour accueillir la page
compressée. Si un tel espace existe, la page est enregistrée dans cette zone.

    \paragraph{}Un nœud est ensuite ajouté dans un \texttt{radix tree} du
kernel pour enregistrer le correspondance entre une \texttt{swp\_entry\_t} et
la page compressée. A cet instant, la page est considérée comme étant présente
dans le ZRLE cache.

    \paragraph{}Etant donné que les pages sont interceptées dans la fonction
\texttt{swap\_writepage()}, une \texttt{swp\_entry\_t} a été déjà allouée pour
chacune d'elle. Malheureusement, cela implique qu'un espace a été alloué sur
la \texttt{swap device} alors qu'il est inutilisé. Cependant, il s'agit de la
solution la plus simple que nous avons pu trouver. Nous avons d'abord envisagé
d'intercepter la page dans \texttt{shrink\_page\_list()}, avant qu'un secteur
de la swap ne soit alloué, mais cela s'avère beaucoup trop complexe (notamment,
tout \texttt{pte\_t} pourrait pointer vers une entrée du cache en plus d'une page
physique ou d'une entrée d'une \texttt{swap device}, et ce cas devrait être
envisagé à chaque endroit où un \texttt{pte\_t} est accédé).

    \paragraph{}Si la page ne peut pas être suffisamment compressée, ou si
aucun espace n'est disponible dans le cache, la page est transmise à la
\texttt{swap device} comme dans le noyau non modifié.

 \subsection{Recopie des bins du cache sur le disque}

    \paragraph{}Nous ne sommes malheureusement pas parvenus à implémenter le
transfert des "bins" remplis sur le disque.

 \subsection{Récupération d'une page}

    \paragraph{}Notre implémentation intercepte les requêtes de récupération
des pages de swap dans la fonction \texttt{swap\_writepage()}. Pour chaque
requête, nous recherchons si une correspondance à la \texttt{swp\_entry\_t}
existe dans l'arbre de recherche du cache.

    \paragraph{}Si une telle correspondance existe, le contenu compressé est
restauré dans la page, le contenu du cache désalloué et l'accès à la 
\texttt{swap device} évité.

    \paragraph{}Si aucune correspondance n'existe, alors la page se trouve sur
la \texttt{swap device} et le noyau reprend son cours normal d'exécution.

 \subsection{Structure du cache}

    \paragraph{}Nous avons organisé notre cache de manière relativement efficace.

    \paragraph{}Chaque "bin" se comporte comme un allocateur dynamique en
utilisant une structuration similaire à celle décrite au cours de cet article :

    \begin{center}
        \textbf{http://jamesgolick.com/2013/5/15/memory-allocators-101.html}
    \end{center}

    \paragraph{}Notre implémentation permet notamment de réduire la
fragmentation induite lorsque des pages compressées sont supprimées du cache de
manière rapide et efficace.

    \paragraph{}Comme dit plus haut, chaque page compressée dans le cache est
répertoriée dans un arbre de recherche à partir de son \texttt{swp\_entry\_t},
pour rendre les interceptions des pages compressées rapides.

    \paragraph{}Pour maintenir la cohérence de notre cache, nous avons envisagé
de protéger les opérations sur celui-ci à l'aide d'un \texttt{spin\_lock}.
Malheureusement, pour une raison qui nous est inconnue, celui-ci semblait
provoquer des dead-locks.

\end{document}