\documentclass[a4paper]{article}
\usepackage[utf8]{inputenc}
\usepackage[english]{babel}
\usepackage[babel=true]{csquotes}
\usepackage{graphicx}
\usepackage{multirow}

\pagestyle{headings}

\title{Programming techniques : Project 4}
\author{Raphaël Javaux and Marien Bourguignon}
\date{}

\begin{document}

\maketitle

   \section{Queue management}

    \paragraph{}Queuing is done as follow :
    \begin{itemize}
        \item If the shower is empty, then anybody can go inside without any
        queuing ;
        \item If the showed is used by someone of the opposite sex, then the
        person will wait until the shower is empty ;
        \item If the shower is used by someone of the same sex and if there is
        nobody of the opposite sex waiting outside, then the person is allowed 
        to immediately join the shower ;
        \item If the shower is used by someone of the same sex but if there is 
        at least one person of the opposite sex waiting outside, then the person
        will be en-queued after people of the opposite sex.
    \end{itemize}
    The last condition is there to ensure that someone of sex A who arrives
    after someone of sex B doesn't bypass this last, avoiding any
    starvation, while other rules enforce mutual exclusion.\newline
    Once the shower is empty again, every person of the opposite sex, if any, is
    allowed to enter the shower. Waking up a group of people is done using a
    conditional variable.

    \paragraph{}\textbf{We can easily prove that our queuing algorithm doesn't
    give any way for a person to be kept in the queue for more than the double
    of the shower length time.}

  \section{Results}

    \paragraph{}We've tested our queuing algorithm of two different ordering.

    \paragraph{}In the first case, both sexes are already grouped in two
    distinct groups, without any interlacing. This is the best case of our
    algorithm as there will only be two shower sessions. As expected, the first
    group of people doesn't wait to enter as a whole in the shower while the
    second group must wait until the last person of the first group has leaved
    the shower. \textbf{Giving $N$ people of sex $A$ and $M$ people of sex $B$,
    the average waiting time will be $\frac{M}{N + M}$ times the defined shower
    length time} as only people of group $B$ will have to wait outside the shower.

    \paragraph{}In the second test case, people enter the shower in a shuffled
    ordering. This is not far away from the worst case scenario of our algorithm
    as the first person of sex B will prohibit anybody of sex A to enter the shower,
    even if the shower is currently used by someone of sex A. Thus there will be
    three shower sessions. \textbf{Giving $N$ people of sex $A$ and $M$ people
    of sex $B$, the worst average waiting time is given by the following
    formula :}

    \[
        max\left(\frac{M + 2(N - 1)}{N + M}, \frac{N + 2(M - 1)}{N + M}\right)
    \]
    as in this scenario, only one person of either group A or group B
    will be allowed to enter the shower without waiting (because of being
    immediately followed by a person of the opposite group) whereas every
    remaining person of his group will have to wait for two shower length
    times.\newline
    \textbf{Experiments confirm this theory by giving an average waiting time of
    between 1.35 and 1.46 time the shower length time} while the theoretical
    maximum of the preceding formula is 1.5 for the same values of $N$ and $M$.

\end{document}